\documentclass{article}

\usepackage{amsmath, amsthm, amssymb, amsfonts}
\usepackage{thmtools}
\usepackage{graphicx}
\usepackage{setspace}
\usepackage{geometry}
\usepackage{float}
\usepackage{hyperref}
\usepackage[utf8]{inputenc}
\usepackage[english]{babel}
\usepackage{framed}
\usepackage[dvipsnames]{xcolor}
\usepackage[most]{tcolorbox}
\usepackage{minted}
\usepackage{enumitem}
\usepackage{booktabs}
\usepackage{amsmath}

\usepackage{indentfirst}

\usepackage[export]{adjustbox} % Align images

\colorlet{LightGray}{White!90!Periwinkle}
\colorlet{LightOrange}{Orange!15}
\colorlet{LightGreen}{Green!15}

\newcommand{\HRule}[1]{\rule{\linewidth}{#1}}

\newtcbtheorem[auto counter,number within=section]{code}{Código}{
  colback=LightOrange!20,
  colframe=LightOrange,
  colbacktitle=LightOrange,
  fonttitle=\bfseries\color{black},
  boxed title style={size=small,colframe=LightOrange},
}{code}

\setstretch{1.2}
\geometry{
  textheight=22.5cm,
  textwidth=13.75cm,
  top=2.5cm,
  headheight=12pt,
  headsep=25pt,
  footskip=30pt
}

\DeclareMathOperator*{\argmax}{argmax}
% ------------------------------------------------------------------------------

\begin{document}

% ------------------------------------------------------------------------------
% Cover Page and ToC
% ------------------------------------------------------------------------------
\begin{center}
  \begin{figure}
    \includegraphics[scale = 0.3, left]{img/IST_A.eps} % IST logo
    \end{figure}
  \LARGE{ \normalsize \textsc{} \\
  [2.0cm] 
  \LARGE{ \LARGE \textsc{Aprendizagem}} \\
  [1cm]
  \LARGE{ \LARGE \textsc{LEIC IST-UL}} \\
  [1cm]
  \HRule{1.5pt} \\
  [0.4cm]
  \LARGE \textbf{\uppercase{Relatório - Homework 1}}
  \HRule{1.5pt}
  \\ [2.5cm]
  }
\end{center}

\begin{flushleft}
  \textbf{\LARGE Grupo 10:}
\end{flushleft}

\begin{center}
  \begin{minipage}{0.7\textwidth}
      \begin{flushleft}
        \large Gabriel Ferreira \\
        \large  Irell Zane
      \end{flushleft}
  \end{minipage}%
  \begin{minipage}{0.3\textwidth}
      \begin{flushright}
        \large 107030\\
        \large 107161
      \end{flushright}
  \end{minipage}
\end{center}

\begin{center}
  \vspace{4cm}
  \date \large \bf  2024/2025 -- 1st Semester, P1
\end{center}

\setcounter{page}{0}
\thispagestyle{empty}
\renewcommand{\thesection}{\Roman{section}}

\newpage

% ------------------------------------------------------------------------------
% Content
% ------------------------------------------------------------------------------



\large{\textbf{Part I}: Pen and paper}\normalsize

\begin{enumerate}[leftmargin=\labelsep]
\item Question summary can go here.
    \begin{enumerate}
    \item Place your solution. Math can be entered using the equation
    environment like this
    \begin{equation}
        \vec{\mathbf{r}} = \vec{\mathbf{r}}_{0} + \vec{\mathbf{v}}_{0}t + \frac{1}{2}\vec{\mathbf{a}}t^{2}
    \end{equation}
    If you then where working in say the $x$-direction and had some numbers % A percent sign allows you to comment.
    %The dollar signs around something in a line of text is for "in-line math"
    \begin{equation}
    \begin{array}{r@{~=~}l}
    x & x_{0} + v_{x0}t + \frac{1}{2}a_{x}t^{2} \\ [2ex]
    & 1.2~\text{m} + (4.0~\text{m/s})(3.0~\text{s}) + \frac{1}{2}(-1.0~\text{m/s}^{2})(3.0~\text{s})^{2}\\ [2ex]
    & \boxed{8.7~\text{m}}
    \end{array}
    \end{equation}

    \item When you get to the next part, you can add a \verb"\item" to get the appropriate label. Also,
    if you don't like all the equation numbers, you can use the following to have the equation with
    no number
    \begin{equation*}
    \sum\vec{\mathbf{F}} = m\vec{\mathbf{a}}
    \end{equation*}

    \item For more details on putting math into {\LaTeX} documents you can see 
    \href{https://www.overleaf.com/learn/latex/Mathematical_expressions}{this page on Overleaf}.
    \end{enumerate}

\item We you get to the next problem, you can end the enumerate for the parts of the previous problem and then add another item.
    \begin{enumerate}
    \item Use a nested enumerate environment to label the parts of the next problem.
    \item For a quick and broad overview of how to create documents in {\LaTeX} see 
    \href{https://www.overleaf.com/learn/latex/Learn_LaTeX_in_30_minutes}{this quick tutorial from Overleaf}.
    \end{enumerate}

\item \begin{enumerate}
    \item \textbf{Dataset}:
    \begin{table}[h]
    \centering
    \begin{tabular}{ccccc}
    \toprule
    $x$ & $y_1$ & $y_2$ & $y_3$ & Class \\
    \midrule
    1 & A & 0 & 1.1 & P \\
    2 & B & 1 & 0.8 & P \\
    3 & A & 1 & 0.5 & P \\
    4 & A & 0 & 0.9 & P \\
    5 & B & 0 & 1.0 & N \\
    6 & B & 0 & 0.9 & N \\
    7 & A & 1 & 1.2 & N \\
    8 & B & 1 & 0.9 & N \\
    9 & B & 0 & 0.8 & P \\
    \bottomrule
    \end{tabular}
    \caption{Observed Values}
    \end{table}

    \item \textbf{Priors}:
    \begin{itemize}
        \item For class Positive (P):
        \[
        P(P) = \frac{5}{9}
        \]
        \item For class Negative (N):
        \[
        P(N) = \frac{4}{9}
        \]
    \end{itemize}

    \item \textbf{Class-conditional Probabilities}:
    \begin{itemize}
        \item Calculate the probabilities of the variable set \(\{y_1, y_2\}\) given each class.
        \item The values are calculated as follows:
        \begin{align*}
            P(y_1 = A, y_2 = 0) &= \frac{2}{9} \\
            P(y_1 = A, y_2 = 1) &= \frac{2}{9} \\
            P(y_1 = B, y_2 = 0) &= \frac{2}{9} \\
            P(y_1 = B, y_2 = 1) &= \frac{3}{9} \\
            P(y_1 = A, y_2 = 0 | P) &= \frac{2}{5} \\
            P(y_1 = A, y_2 = 1 | P) &= \frac{1}{5} \\
            P(y_1 = B, y_2 = 0 | P) &= \frac{1}{5} \\
            P(y_1 = B, y_2 = 1 | P) &= \frac{1}{5} \\
            P(y_1 = A, y_2 = 0 | N) &= 0 \\
            P(y_1 = A, y_2 = 1 | N) &= \frac{1}{4} \\
            P(y_1 = B, y_2 = 0 | N) &= \frac{2}{4} \\
            P(y_1 = B, y_2 = 1 | N) &= \frac{1}{4}
        \end{align*}
    \end{itemize}

    \item \textbf{Mean and Standard Deviation of \(y_3\)}:
    \begin{itemize}
        \item For all instances:    
        \begin{align*}
            \mu_{y_3} &= \frac{1.1 + 0.8 + 0.5 + 0.9 + 0.8 + 1.0 + 0.9 + 1.2 + 0.9}{9} \\ 
                      &= 0.9
        \end{align*}
        
        \begin{align*}
            \sigma_{y_3} &= \sqrt{\frac{(1.1 - 0.9)^2 + (0.8 - 0.9)^2 + (0.5 - 0.9)^2 + (0.9 - 0.9)^2...}{9}} \\
                          &\approx 0.2
        \end{align*}
        
        \item For Positive (P):
        \begin{align*}
            \mu_{y_3,P} &= \frac{1.1 + 0.8 + 0.5 + 0.9 + 0.8}{5} = 0.82 \\
            \sigma_{y_3,P} &= \sqrt{\frac{(1.1 - 0.82)^2 + (0.8 - 0.82)^2 + (0.5 - 0.82)^2 + ...}{5}} \\
                          &\approx 0.217
        \end{align*}
        
        \item For Negative (N):
        \begin{align*}
            \mu_{y_3,N} &= \frac{1.0 + 0.9 + 1.2 + 0.9}{4} = 1.0 \\
            \sigma_{y_3,N} &= \sqrt{\frac{(1.0 - 1.0)^2 + (0.9 - 1.0)^2 + (1.2 - 1.0)^2 + (0.9 - 1.0)^2}{4}} \\
                          &\approx 0.1414
        \end{align*}

    \end{itemize}

    \item \textbf{Prediction of Class}:
    \begin{itemize}
        \item To predict the class of a new instance \((y_1, y_2, y_3)\), we calculate the probability for both classes (Positive and Negative) and choose the class with the higher probability:
        \[
        \text{Predicted Class} = \argmax_h P(h | y_1, y_2, y_3)
        \]
    \end{itemize}
    
    \begin{align*}
    \text{where } P(h | y_1, y_2, y_3) &= \frac{P(y_1, y_2, y_3 | h) \cdot P(h)}{P(y_1, y_2, y_3)} \\[10pt]
    &= \frac{P(y_1, y_2 | h) \cdot P(y_3 | h) \cdot P(h)}{P(y_1, y_2) \cdot P(y_3)} \\[10pt]
    \text{where } P(y_3 | h) &= \frac{1}{\sigma_h \sqrt{2\pi}} \exp\left(-\frac{(y_3 - \mu_h)^2}{2\sigma_h^2}\right) \\[10pt]
    \text{and } P(y_3) &= \frac{1}{0.2 \sqrt{2\pi}} \exp\left(-\frac{(y_3 - 0.9)^2}{2 \cdot 0.2^2}\right)
    \end{align*}

\end{enumerate}

\item \begin{enumerate} 
    \item For instance (A, 1, 0.8):


    For class \( P \):
    \begin{align*}
    P(y_3=0.8 | P) &= \frac{1}{0.217 \sqrt{2\pi}} \exp\left(-\frac{(0.8 - 0.82)^2}{2 \cdot 0.217^2}\right) \\[10pt]
    &= 1.83\\
    P(y_3=0.8) &= \frac{1}{0.2 \sqrt{2\pi}} \exp\left(-\frac{(0.8 - 0.9)^2}{2 \cdot 0.2^2}\right) \\[10pt]
    &= 1.76\\
    P(P | y_1=A, y_2=1, y_3=0.8) &= \frac{\frac{5}{9} \cdot \frac{1}{5} \cdot 1.83}{\frac{2}{9} \cdot 1.76} \\[10pt]
    &\approx 0.519
    \end{align*}

    
    For class N:
    \begin{align*}
    P(y_3=0.8 | N) &= \frac{1}{0.1414 \sqrt{2\pi}} \exp\left(-\frac{(0.8 - 1.0)^2}{2 \cdot 0.1414^2}\right) \\[10pt]
    &= 1.038\\
    P(N | y_1=A, y_2=1, y_3=0.8) &= \frac{\frac{4}{9} \cdot \frac{1}{4} \cdot 1.038}{\frac{2}{9} \cdot 1.76} \\[10pt]
    &\approx 0.295
    \end{align*}
    
    Since $0.519 > 0.295$, the predicted class is P.
    
    \item For instance (B, 1, 1):
    
    For class \( P \):
    \begin{align*}
    P(y_3=1 | P) &= \frac{1}{0.217 \sqrt{2\pi}} \exp\left(-\frac{(1 - 0.82)^2}{2 \cdot 0.217^2}\right) \\[10pt]
    &=1.304 \\
    P(y_3=1) &= \frac{1}{0.2 \sqrt{2\pi}} \exp\left(-\frac{(1 - 0.9)^2}{2 \cdot 0.2^2}\right) \\[10pt]
    &= 1.76\\
    P(P | y_1=B, y_2=1, y_3=1) &= \frac{\frac{5}{9} \cdot \frac{1}{5} \cdot 1.304}{\frac{3}{9} \cdot 1.76} \\[10pt]
    &\approx 0.247 
    \end{align*}

    
    For class N:
    \begin{align*}
    P(y_3=1 | N) &= \frac{1}{0.1414 \sqrt{2\pi}} \exp\left(-\frac{(1 - 1.0)^2}{2 \cdot 0.1414^2}\right) \\[10pt]
    &= 2.82\\
    P(N | y_1=B, y_2=1, y_3=1) &= \frac{\frac{4}{9} \cdot \frac{1}{4} \cdot 2.82}{\frac{3}{9} \cdot 1.76} \\[10pt]
    &\approx 0.534
    \end{align*}
    
    Since $0.247 < 0.534$, the predicted class is N.
    
    \item For instance (B, 0, 0.9):
    
    For class \( P \):
    \begin{align*}
    P(y_3=0.9 | P) &= \frac{1}{0.217 \sqrt{2\pi}} \exp\left(-\frac{(0.9 - 0.82)^2}{2 \cdot 0.217^2}\right) \\[10pt]
    &= 1.72\\
    P(y_3=0.9) &= \frac{1}{0.2 \sqrt{2\pi}} \exp\left(-\frac{(0.9 - 0.9)^2}{2 \cdot 0.2^2}\right) \\[10pt]
    &= 1.99\\
    P(P | y_1=B, y_2=0, y_3=0.9) &= \frac{\frac{5}{9} \cdot \frac{1}{5} \cdot 1.72}{\frac{2}{9} \cdot 1.99} \\[10pt]
    &\approx 0.432
    \end{align*}

    
    For class N:
    \begin{align*}
    P(y_3=0.9 | N) &= \frac{1}{0.1414 \sqrt{2\pi}} \exp\left(-\frac{(0.9 - 1.0)^2}{2 \cdot 0.1414^2}\right) \\[10pt]
    &= 2.20\\
    P(N | y_1=B, y_2=0, y_3=0.9) &= \frac{\frac{4}{9} \cdot \frac{2}{4} \cdot 2.20}{\frac{2}{9} \cdot 1.99} \\[10pt]
    &\approx 1.106
    \end{align*}
    
    Since $0.432 < 1.106$, the predicted class is P.
\end{enumerate}

\end{enumerate}

\large{\textbf{Part II}: Programming}\normalsize

\begin{enumerate}[leftmargin=\labelsep,resume]
\item Solution to the programming questions here.
\end{enumerate}

\vskip 1cm
\textbf{End note}: do not forget to also submit your Jupyter notebook

\newpage

% ----------------------------------------------------------------------
% Cover
% ----------------------------------------------------------------------

\end{document}

